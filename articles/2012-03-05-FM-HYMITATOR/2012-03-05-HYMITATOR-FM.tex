\documentclass{llncs}
\sloppy

%%%%%%%%%%%%%%%%%%%%%%%%%%%%%%%%%%%%%%%%%%%%%%%%%%%%%%%%%%%%
% PACKAGES
%%%%%%%%%%%%%%%%%%%%%%%%%%%%%%%%%%%%%%%%%%%%%%%%%%%%%%%%%%%%
\usepackage[utf8x]{inputenc}

% SYMBOLES
\usepackage{amsmath, amssymb, url} % amsthm, 

% ALGORITHMES
\usepackage[ruled,vlined,linesnumbered]{algorithm2e}



%%%%%%%%%%%%%%%%%%%%%%%%%%%%%%%%%%%%%%%%%%%%%%%%%%%%%%%%%%%%
% TIKZ
%%%%%%%%%%%%%%%%%%%%%%%%%%%%%%%%%%%%%%%%%%%%%%%%%%%%%%%%%%%%
\usepackage{pgf}
\usepackage{tikz}

% Couleurs
\definecolor{turquoise}{rgb}{0 0.41 0.41}
\definecolor{rouge}{rgb}{0.79 0.0 0.1}
\definecolor{vert}{rgb}{0.15 0.4 0.1}
\definecolor{mauve}{rgb}{0.6 0.4 0.8}
\definecolor{violet}{rgb}{0.58 0. 0.41}
\definecolor{orange}{rgb}{0.8 0.4 0.2}
\definecolor{bleu}{rgb}{0.39, 0.58, 0.93}
\definecolor{gris}{rgb}{0.6,0.6,0.6}
\definecolor{grisfonce}{rgb}{0.4, 0.4, 0.4}
\definecolor{grispale}{rgb}{0.9, 0.9, 0.9}
% NOIR ET BLANC
\definecolor{vertfonce}{rgb}{0.0, 0, 0.0}
\definecolor{rougefonce}{rgb}{0, 0.0, 0.0}
\definecolor{rougeclair}{rgb}{0.6, 0.6, 0.6} %{red!50!white}
\definecolor{bleufonce}{rgb}{0.2, 0.2, 0.2} %blue!80!black
\definecolor{bleutresfonce}{rgb}{0.1, 0.1, 0.1} %blue!50!black
% \definecolor{vertfonce}{rgb}{0.0, 0.5, 0.0}
% \definecolor{rougefonce}{rgb}{1, 0.0, 0.0}
% \definecolor{rougeclair}{rgb}{1, 0.5, 0.5} %{red!50!white}
% \definecolor{bleufonce}{rgb}{0, 0, 0.8} %blue!80!black
% \definecolor{bleutresfonce}{rgb}{0, 0, 0.5} %blue!50!black


% Jeu de couleurs pales
% NOIR ET BLANC
\definecolor{cpale1}{rgb}{0.6, 0.6, 0.6}
\definecolor{cpale2}{rgb}{0.9, 0.9, 0.9}
% \definecolor{cpale1}{rgb}{1, 0.6, 0.6}
% \definecolor{cpale2}{rgb}{0.6, 1, 0.6}
\definecolor{cpale3}{rgb}{0.6, 0.6, 1}
\definecolor{cpale4}{rgb}{1, 0.6, 1}
\definecolor{cpale5}{rgb}{1, 1, 0.6}
\definecolor{cpale6}{rgb}{0.6, 1, 1}
\definecolor{cpale7}{rgb}{0.95, 0.65, 0.25}
\definecolor{cpale8}{rgb}{0.75, 0.45, 1}
\definecolor{cpale9}{rgb}{0.5, 1, 0.75}
\definecolor{cpale10}{rgb}{0.8, 0.7, 0.6}
\definecolor{cpale11}{rgb}{0.6, 0.7, 0.8}
\definecolor{cpale12}{rgb}{0.2, 0.5, 0.9}
\definecolor{cpale13}{rgb}{0.5, 0.9, 0.2}
\definecolor{cpale14}{rgb}{0.9, 0.2, 0.5}
\definecolor{cpale15}{rgb}{0.7, 0.7, 0.7}
\definecolor{cpale16}{rgb}{0.8, 0.8, 0.5}



%%%%%%%%%%%%%%%%%%%%%%%%%%%%%%%%%%%%%%%%%%%%%%%%%%%%%%%%%%%%
% CONSTANTES
%%%%%%%%%%%%%%%%%%%%%%%%%%%%%%%%%%%%%%%%%%%%%%%%%%%%%%%%%%%%

%-%-%-%-%-%-%-%-%-%-%-%-%-%-%-%-%-%-%-%-%-%-%-%-%-%-%-%-%-%
% CONSTANTES MATHEMATIQUES
%-%-%-%-%-%-%-%-%-%-%-%-%-%-%-%-%-%-%-%-%-%-%-%-%-%-%-%-%-%
% Booleens
\newcommand{\false}{{\tt false}}
\newcommand{\true}{{\tt true}}

% Ensembles
\newcommand{\AP}{\mathit{AP}}
\newcommand{\grandn}{{\mathbb N}}
\newcommand{\grandq}{{\mathbb Q}}
\newcommand{\grandqplus}{{\mathbb Q}_{\geq 0}}
\newcommand{\grandr}{{\mathbb R}}
\newcommand{\grandrplus}{\grandr_{\geq 0}}
\newcommand{\grandz}{{\mathbb Z}}
\newcommand{\setX}{\mathcal{K}_{\Clock}}
\newcommand{\setP}{\mathcal{K}_{\Param}}
\newcommand{\setXP}{\mathcal{K}_{\Clock \cup \Param}}

% Matrices
\newcommand{\matrixzero}{\mathbf{0}}
\newcommand{\matrixnopath}{\emptyset}

% Noms
\newcommand{\tiling}{\mathit{Tiling}}
\newcommand{\post}{\mathit{Post}}
\newcommand{\Pre}{\mathit{Pre}}
\newcommand{\TS}{\mathit{TS}}
\newcommand{\words}{\mathit{Words}}

% Operateurs
\DeclareMathOperator*{\argmax}{arg\,max}
\DeclareMathOperator*{\argmin}{arg\,min}
\newcommand{\eqdef}{ \overset{\mathit{def}}{=}}


% Symboles
\newcommand{\cro}[1]{\langle #1 \rangle}
\newcommand{\fleche}[1]{\stackrel{#1}{\rightarrow}}
\newcommand{\Fleche}[1]{\stackrel{#1}{\Rightarrow}}
\newcommand{\prefix}[2]{ |#1|_{#2} }
\newcommand{\steps}[0]{ {\rightarrow} }
\newcommand{\Steps}[0]{ {\Rightarrow} }
\newcommand{\timelaps}[1]{#1^\uparrow}
\newcommand{\trace}[1]{ \mathbin{<}#1 \mathbin{>}}
\newcommand{\wpi}[1]{\mathbin{<}#1\mathbin{>}}

% Temporal logics
\newcommand{\tlAlways}{\Box}
\newcommand{\tlEvt}{\Diamond}
\newcommand{\tlExists}{\exists}
\newcommand{\tlForall}{\forall}
\newcommand{\tlNext}{\bigcirc} % X
% \newcommand{\tlNext}{\circle}
\newcommand{\tlUntil}{\cup}

% Unites
\newcommand{\micros}{\mathit{\mu\,s}}
\newcommand{\millis}{\mathit{m\,s}}
\newcommand{\nanos}{ns}
\newcommand{\picos}{ps}


% Variables
\newcommand{\A}{\mathcal{A}}
\newcommand{\Avar}{\A_\mathit{var}}
\newcommand{\Clock}{X} % set of clocks
\newcommand{\clock}{x} % clock
\newcommand{\clockval}{w} % clock valuation
\newcommand{\enabled}{\mathit{enabled}}
\newcommand{\events}{E} % a possible set of events
\newcommand{\EventsSet}{\Sigma} % all possible events
\newcommand{\formule}{\varphi}
\newcommand{\Formule}{\Phi}
% \newcommand{\Kinit}{K_\mathit{init}} % initial constraint on the parameters
\newcommand{\LTS}{\mathcal{L}}
\newcommand{\Param}{P} % set of parameters (P / Y)
\newcommand{\param}{p} % parameter (p / y)
\newcommand{\py}{\pi} % parameter valuation
\newcommand{\Py}{\Pi} % set of consistent parameter valuations
\newcommand{\sinit}{s_\mathit{init}} % initial set of states
\newcommand{\Slast}{S_\mathit{last}}



\newcommand{\Ko}{K_0}
\newcommand{\pio}{\pi_0}
\newcommand{\piprime}{\pi}
\newcommand{\To}{T_0}
\newcommand{\Tprime}{T}


% \newcommand{\compyes}{\textcolor{green}{yes}}
\newcommand{\compyes}{$\textcolor{vertfonce}{\mathbf{\surd}}$}
% \newcommand{\compno}{\textcolor{red}{no}}
\newcommand{\compno}{$\textcolor{rougefonce}{\mathbf{\times}}$}


% Pour MDP
\newcommand{\we}{w}
\newcommand{\We}{W}
\newcommand{\va}{v}
\newcommand{\Va}{V}
\newcommand{\prob}{\mathit{Prob}}

%-%-%-%-%-%-%-%-%-%-%-%-%-%-%-%-%-%-%-%-%-%-%-%-%-%-%-%-%-%
% PARAMETRES
%-%-%-%-%-%-%-%-%-%-%-%-%-%-%-%-%-%-%-%-%-%-%-%-%-%-%-%-%-%
% PARAMETRES BRP
\newcommand{\brpMAX}{\mathit{MAX}}
\newcommand{\brpSYNC}{\mathit{SYNC}}
\newcommand{\brpTD}{\mathit{TD}}
\newcommand{\brpTR}{\mathit{TR}}
\newcommand{\brpTun}{\mathit{T1}}

% PARAMETRES CSMACD
\newcommand{\slot}{\mathit{slot}}

% PARAMETRES FLIP FLOP
\newcommand{\THold}{T_{\mathit{Hold}}}
\newcommand{\TSetup}{T_{\mathit{Setup}}}
\newcommand{\THI}{T_{\mathit{HI}}}
\newcommand{\TLO}{T_{\mathit{LO}}}
\newcommand{\TCKQ}{T_{ \mathit{CK} \rightarrow Q}}

% PARAMETRES RCP
\newcommand{\rcpFMax}{\mathit{f\_max}} % \mathit{rc\_fast\_max}
\newcommand{\rcpFMin}{\mathit{f\_min}} % \mathit{rc\_fast\_min}
\newcommand{\rcpSMax}{\mathit{s\_max}} % \mathit{rc\_slow\_max}
\newcommand{\rcpSMin}{\mathit{s\_min}} % \mathit{rc\_slow\_min}
\newcommand{\rcpD}{\mathit{delay}}

% PARAMETRES SIMOP
\newcommand{\COMct}{\mathit{COMct}}
\newcommand{\COMd}{\mathit{COMd}}
\newcommand{\NETd}{\mathit{NETd}}
\newcommand{\PLCct}{\mathit{PLCct}}
\newcommand{\PLCmtt}{\mathit{PLCmtt}}
\newcommand{\RIOd}{\mathit{RIOd}}
\newcommand{\SIGmrt}{\mathit{SIGmrt}}

% PARAMETRES SPSMALL
\newcommand{\tSetupA}{t_\mathit{setup}^A}
\newcommand{\tSetupD}{t_\mathit{setup}^D}
\newcommand{\tSetupWen}{t_\mathit{setup}^\mathit{WEN}}


% PARAMETRES WLAN
\newcommand{\pACK}{\mathit{ACK}}
\newcommand{\pACKTO}{\mathit{ACKTO}}
\newcommand{\pASLOTTIME}{\mathit{ASLOTTIME}}
\newcommand{\pBOFF}{\mathit{BOFF}}
\newcommand{\pDIFS}{\mathit{DIFS}}
\newcommand{\pMAXCOL}{\mathit{MAXCOL}}
\newcommand{\pSIFS}{\mathit{SIFS}}
\newcommand{\pTRANSTIMEMIN}{\mathit{TTMIN}}
\newcommand{\pTRANSTIMEMAX}{\mathit{TTMAX}}
\newcommand{\pVULN}{\mathit{VULN}}


%-%-%-%-%-%-%-%-%-%-%-%-%-%-%-%-%-%-%-%-%-%-%-%-%-%-%-%-%-%
% ALGORITHMES
%-%-%-%-%-%-%-%-%-%-%-%-%-%-%-%-%-%-%-%-%-%-%-%-%-%-%-%-%-%
% Algorithmes PTA
\newcommand{\IM}{\mathit{IM}}
\newcommand{\IMincl}{\IM_\subseteq}
\newcommand{\IMK}{\IM^K}
\newcommand{\IMunion}{\IM^\cup}
\newcommand{\IMinclK}{\IM^K_\subseteq}
\newcommand{\IMinclunion}{\IM^\cup_\subseteq}
\newcommand{\IMotf}{\IM_\mathit{otf}}
\newcommand{\BC}{\mathit{BC}}
\newcommand{\BCincl}{\BC_\subseteq}
\newcommand{\BCK}{\BC^K}
\newcommand{\BCunion}{\BC^\cup}
\newcommand{\BCinclK}{\BC^K_\subseteq}
\newcommand{\BCinclunion}{\BC^\cup_\subseteq}


%-%-%-%-%-%-%-%-%-%-%-%-%-%-%-%-%-%-%-%-%-%-%-%-%-%-%-%-%-%
% CONSTANTES DE CHAINES
%-%-%-%-%-%-%-%-%-%-%-%-%-%-%-%-%-%-%-%-%-%-%-%-%-%-%-%-%-%

% Langue
\newcommand{\absurde}{\emph{reductio ad absurdum}}

% Divers (projets et technique)

\newcommand{\lipsix}{LIP\,6}
\newcommand{\parissix}{Universit\'e Pierre et Marie Curie}
\newcommand{\simop}{SIMOP}
\newcommand{\spsmall}{SPSMALL}
\newcommand{\stm}{ST-Microelectronics}
\newcommand{\therac}{Therac-25}
\newcommand{\valmem}{VALMEM}

% Outils
\newcommand{\apron}{\textsc{Apron}}
\newcommand{\gdot}{\textsc{dot}}
\newcommand{\graphviz}{Graphviz}
\newcommand{\hytech}{{\sc HyTech}}
\newcommand{\imitator}{\textsc{Imitator}}
\newcommand{\imitatordeux}{\textsc{Imitator}\,II}
\newcommand{\hymitator}{\textsc{Hymitator}}
\newcommand{\kronos}{\textsc{Kronos}}
\newcommand{\nusmv}{NuSMV}
\newcommand{\ocaml}{OCaml}
\newcommand{\phaver}{PHAVer}
\newcommand{\plot}{gnuplot}
\newcommand{\polka}{NewPolka}
\newcommand{\prism}{\textsc{Prism}}
\newcommand{\python}{Python}
\newcommand{\red}{RED}
\newcommand{\romeo}{Rom\'eo}
\newcommand{\smv}{SMV}
\newcommand{\spin}{Spin}
\newcommand{\tina}{TINA}
\newcommand{\trex}{\textsc{TReX}}
\newcommand{\uppaal}{\textsc{Uppaal}}
\newcommand{\vhdlta}{\textsc{Vhdl2Ta}}






%%%%%%%%%%%%%%%%%%%%%%%%%%%%%%%%%%%%%%%%%%%%%%%%%%%%%%%%%%%%
% FORMATING
%%%%%%%%%%%%%%%%%%%%%%%%%%%%%%%%%%%%%%%%%%%%%%%%%%%%%%%%%%%%

% CODE INTEGRE AU TEXTE
\newcommand{\code}[1]{\textbf{\texttt{#1}}}

% COMMENTAIRE DANS UN ALGORITHME
\newcommand{\comalgo}[1]{\textcolor{grisfonce}{\tcp*{ #1 }}}


% COMMENTAIRES DANS UN PARAGRAPHE
\newcommand{\commentaire}[1]{\textcolor{red}{\textbf{$\Leftarrow$  #1 $\Rightarrow$}}}
% \newcommand{\commentaire}[1]{}

% REDEFINITIONS DES PARAGRAPHES
\newcommand{\paragraphe}[1]{\paragraph{#1.}}
% \newcommand{\paragraphe}[1]{\subsubsection{#1.}}


% Title
\title{Efficient Analysis of Parametric Hybrid Systems using HYMITATOR}
% \thanks{A long version of this paper is available in~\cite{AS11}, with full preliminary definitions, and all detaild algorithms and proofs.}
\author{\'Etienne Andr\'e$^1$, Laurent Fribourg$^2$, Ulrich K\"uhne$^3$ and Romain Soulat$^2$}
\institute{$^1$LIPN, CNRS UMR 7030, Université Paris 13, France \\
$^2$ENS Cachan, CNRS, LSV, UMR8643 \\
$^3$Universit\"at Bremen, Germany}


\begin{document}

\maketitle

\begin{abstract}
	\commentaire{to do}
\end{abstract}

\keywords{Real-Time Systems, Hybrid Automata, Verification, Parameter Synthesis}

\commentaire{Version avec commentaires}


%%%%%%%%%%%%%%%%%%%%%%%%%%%%%%%%%%%%%%%%%%%%%%%%%%%%%%%%%%%%
%%%%%%%%%%%%%%%%%%%%%%%%%%%%%%%%%%%%%%%%%%%%%%%%%%%%%%%%%%%%
\section{Motivation}
%%%%%%%%%%%%%%%%%%%%%%%%%%%%%%%%%%%%%%%%%%%%%%%%%%%%%%%%%%%%
%%%%%%%%%%%%%%%%%%%%%%%%%%%%%%%%%%%%%%%%%%%%%%%%%%%%%%%%%%%%

\commentaire{short introduction}

In~\cite{acef09}, we proposed the inverse method for Timed Automata, a subclass of hybrid systems whose variables (named clocks) all have constant rates equal to~1.
Different from CEGAR-based methods, this original semi-algorithm for parameter synthesis is based on a ``good'' parameter valuation~$\pio$ instead of a set of ``bad'' states.
$\IM$ synthesizes a constraint~$\Ko$ on the parameters such that, for all parameter valuation~$\piprime$ satisfying $\Ko$, the trace set, i.e., the discrete behavior, of~$\A$ under~$\piprime$ is the same as for~$\A$ under~$\pio$.
This preserves in particular linear time properties, and provides the system with a criterion of robustness.
By iterating the inverse method on all integer points within a bounded reference parameter domain, we get a set of constraints (``tiles'') such that, for every point in each such constraint, the time-abstract behavior is the same: this gives a behavioral cartography of the system~\cite{af10}.

A basic implementation named \imitator{} (for \emph{Inverse Method for Inferring Time AbstracT behaviOR}) has first been proposed%in~\cite{and09}
, under the form of a \python{} script calling \hytech{}~\cite{hhw97}.
The tool has then been entirely rewritten in \imitatordeux{}~\cite{and10}, under the form of a standalone \ocaml{} program making use of the Parma Polyhedra Library (PPL)~\cite{bhz08}.
A number of case studies containing up to 60 timing parameters could be efficiently verified in the purely timed framework.

The inverse method and the behavioral cartography have then been extended to hybrid systems in~\cite{FK11}, and implemented in a prototype ``fork'' of \imitatordeux{}.
\commentaire{small description to add}
We present in this paper \hymitator{}, an extension of that prototype, performing parameter synthesis on hybrid systems.





\commentaire{Laurent a dit :}
- le principe de la methode inverse pour les TA a été implémenté de façon basique (IMITATOR 1 [ICTAC09])
puis de façon sophistiquée (IMITATOR 2 [Infinity10])

- la méthode a été étendue pour les HA et implémentée de façon basique (Imitator 3 [RP11])

- il s'agit "ici" de décrire une implémentation sophistiquée (intégrant notamment des extensions
du merging des TA  à la  [NFM 12] aux HA)


%%%%%%%%%%%%%%%%%%%%%%%%%%%%%%%%%%%%%%%%%%%%%%%%%%%%%%%%%%%%
%%%%%%%%%%%%%%%%%%%%%%%%%%%%%%%%%%%%%%%%%%%%%%%%%%%%%%%%%%%%
\section{Architecture and Features}
%%%%%%%%%%%%%%%%%%%%%%%%%%%%%%%%%%%%%%%%%%%%%%%%%%%%%%%%%%%%
%%%%%%%%%%%%%%%%%%%%%%%%%%%%%%%%%%%%%%%%%%%%%%%%%%%%%%%%%%%%

\hymitator{} takes as input a network of hybrid automata synchronized on shared actions.
The input syntax, inspired by the input syntax of \hytech{}% and \imitator{}
, allows the use of analog variables (i.e., any variable such as temperature, time, etc., evolving in a non-linear manner), rational-valued discrete variables, and parameters (i.e., unknown constants).

The core of the program is written in the object-oriented language OCaml, and interacts with~PPL.
Exact arithmetics is used.
A constraint is output in text format; furthermore, the set of traces computed by the analysis can be output under a graphical form (using \graphviz{}).

%-%-%-%-%-%-%-%-%-%-%-%-%-%-%-%-%-%-%-%-%-%-%-%-%-%-%-%-%-%-%
\begin{figure}[ht!]
	% STYLES
	\tikzstyle{etiquette} = [draw=none, color=black]
	
% \tikzstyle{boite}=[text width=8em, text centered, minimum height=2.5em, rounded corners, very thick]
% \tikzstyle{input}=[boite, draw=green!20!gris, top color=green!50, bottom color=green!10!white]
% \tikzstyle{output}=[boite, draw=red!20!gris, top color=red!50, bottom color=red!5!white]
% \tikzstyle{imitator} = [boite, draw=blue!20!gris, text width=10em, minimum height=6em]

	
	\tikzstyle{boite}=[rectangle, draw=black, rounded corners, thick, draw=blue!40!black, top color=blue!20, bottom color=blue!5!white]
% 	\tikzstyle{imitator} = [boite]
% 	\tikzstyle{apron} = [boite, draw=yellow!20!gris, top color=yellow!70, bottom color=yellow!10!white]
% 	\tikzstyle{polka} = [apron]
% 	\tikzstyle{ppl} = [apron]
	\tikzstyle{dot} = [boite, draw=purple!20!gris, top color=purple!60, bottom color=purple!10!white]

	\tikzstyle{fleche} = [->, draw=black, semithick]
{

\centering

\begin{tikzpicture}[scale=0.6,  =>stealth']

	% Boites
	\draw[boite] (-5, 5.5) rectangle (-2, 6.5);
	\node [etiquette] at (-3.5, 6) {Model};
	
	\draw[boite] (-5, 3.8) rectangle (-2, 5);
	\node [etiquette] at (-3.5, 4.4) {\begin{tabular}{c}Reference\\valuation\end{tabular}};
	
	\draw[boite] (0, 4) rectangle (6, 6.5);
	\node [etiquette] at (3, 5.25) {\large \hymitator{}};
	
	\draw[boite] (1, 2.5) rectangle (5, 3.5);
	\node [etiquette] at (3, 3) {PPL};
	
	\draw[boite] (8, 5.5) rectangle (12, 6.5);
	\node [etiquette] at (10, 6) {Constraint};
	
	\draw[boite] (8, 4) rectangle (12, 5);
	\node [etiquette] at (10, 4.5) {Trace set};

	% Fleches
	\path[fleche] (-2, 6) --++ (2, 0);
	\path[fleche] (-2, 4.4) --++ (2, 0);
	\path[fleche] (3.5, 4) --++ (0, -.5);
	\path[fleche] (2.5, 3.5) --++ (0, .5);
	\path[fleche] (6, 6) --++ (2, 0);
	\path[fleche] (6, 4.5) --++ (2, 0);

\end{tikzpicture}

}

\caption{Architecture of \hymitator{}}
\label{fig:structure}
\end{figure}
%-%-%-%-%-%-%-%-%-%-%-%-%-%-%-%-%-%-%-%-%-%-%-%-%-%-%-%-%-%-%

\hymitator{} implements the following algorithms for hybrid systems:
\begin{description}
	\item[Full reachability analysis] Given a model, it computes the set of symbolic reachable states.
	\item[Inverse method] Given a model and a reference parameter valuation~$\pio$, it computes a constraint on the parameter guaranteeing the same time-abstract behavior as under~$\pio$
	\item[Behavioral cartography] Given a model and a bounded parameter domain for each parameter valuation, it computes a set of constraints and their corresponding trace sets.
\end{description}

\hymitator{} uses several algorithmic optimizations initially developed for \imitator{}.
In particular, the efficient merging presented in~\cite{AFS12} has been successfully extended to the hybrid case: we merge any two states sharing the same discrete part (location and value of the discrete variables) and such that the union of their constraint on the analog variables and parameters is convex.

\commentaire{un mot sur la cartographie puisque, je crois, elle a ete modifiee dans le cas des systemes hybrides}


%%%%%%%%%%%%%%%%%%%%%%%%%%%%%%%%%%%%%%%%%%%%%%%%%%%%%%%%%%%%
%%%%%%%%%%%%%%%%%%%%%%%%%%%%%%%%%%%%%%%%%%%%%%%%%%%%%%%%%%%%
\section{Applications}
%%%%%%%%%%%%%%%%%%%%%%%%%%%%%%%%%%%%%%%%%%%%%%%%%%%%%%%%%%%%
%%%%%%%%%%%%%%%%%%%%%%%%%%%%%%%%%%%%%%%%%%%%%%%%%%%%%%%%%%%%

\commentaire{Laurent a dit :}

- sampled data hybrid systems (cf [RP11])

- synthesis schedulability regions (cf [Cimatti-Palopoli-Ramadian08], [Astrium EADS])

\commentaire{experiences ?}

\commentaire{rapport d'etudes de cas a rediger}



%%%%%%%%%%%%%%%%%%%%%%%%%%%%%%%%%%%%%%%%%%%%%%%%%%%%%%%%%%%%
%%%%%%%%%%%%%%%%%%%%%%%%%%%%%%%%%%%%%%%%%%%%%%%%%%%%%%%%%%%%
\section{Related Work}
%%%%%%%%%%%%%%%%%%%%%%%%%%%%%%%%%%%%%%%%%%%%%%%%%%%%%%%%%%%%
%%%%%%%%%%%%%%%%%%%%%%%%%%%%%%%%%%%%%%%%%%%%%%%%%%%%%%%%%%%%

\commentaire{to do (citer HyTech, Phaver)}

A graphical user interface for the input model is currently under construction, based on a generic platform. \commentaire{utile ?!}






%%%%%%%%%%%%%%%%%%%%%%%%%%%%%%%%%%%%%%%%%%%%%%%%%%%%%%%%%%%%
%%%%%%%%%%%%%%%%%%%%%%%%%%%%%%%%%%%%%%%%%%%%%%%%%%%%%%%%%%%%
%%%%%%%%%%%%%%%%%%%%%%%%%%%%%%%%%%%%%%%%%%%%%%%%%%%%%%%%%%%%
\bibliographystyle{abbrv} % alpha plain
\bibliography{biblioFM}
%%%%%%%%%%%%%%%%%%%%%%%%%%%%%%%%%%%%%%%%%%%%%%%%%%%%%%%%%%%%
%%%%%%%%%%%%%%%%%%%%%%%%%%%%%%%%%%%%%%%%%%%%%%%%%%%%%%%%%%%%
%%%%%%%%%%%%%%%%%%%%%%%%%%%%%%%%%%%%%%%%%%%%%%%%%%%%%%%%%%%%

\newpage


\appendix

\section*{Appendix}

\commentaire{ajouter un trace set en mode fancy}

\commentaire{ajouter un modele en entree avec sa representation graphique (pour decrire la syntaxe)}

\end{document}          
